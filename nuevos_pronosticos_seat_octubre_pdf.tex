\documentclass[ignorenonframetext,]{beamer}
\usetheme{CambridgeUS}
\usecolortheme{sidebartab}
\usefonttheme{structuresmallcapsserif}
\setbeamertemplate{caption}[numbered]
\setbeamertemplate{caption label separator}{: }
\setbeamercolor{caption name}{fg=normal text.fg}
\usepackage{lmodern}
\usepackage{amssymb,amsmath}
\usepackage{ifxetex,ifluatex}
\usepackage{fixltx2e} % provides \textsubscript
\ifnum 0\ifxetex 1\fi\ifluatex 1\fi=0 % if pdftex
  \usepackage[T1]{fontenc}
  \usepackage[utf8]{inputenc}
\else % if luatex or xelatex
  \ifxetex
    \usepackage{mathspec}
  \else
    \usepackage{fontspec}
  \fi
  \defaultfontfeatures{Ligatures=TeX,Scale=MatchLowercase}
  \newcommand{\euro}{€}
\fi
% use upquote if available, for straight quotes in verbatim environments
\IfFileExists{upquote.sty}{\usepackage{upquote}}{}
% use microtype if available
\IfFileExists{microtype.sty}{%
\usepackage{microtype}
\UseMicrotypeSet[protrusion]{basicmath} % disable protrusion for tt fonts
}{}
\newif\ifbibliography
\usepackage{graphicx,grffile}
\makeatletter
\def\maxwidth{\ifdim\Gin@nat@width>\linewidth\linewidth\else\Gin@nat@width\fi}
\def\maxheight{\ifdim\Gin@nat@height>\textheight0.8\textheight\else\Gin@nat@height\fi}
\makeatother
% Scale images if necessary, so that they will not overflow the page
% margins by default, and it is still possible to overwrite the defaults
% using explicit options in \includegraphics[width, height, ...]{}
\setkeys{Gin}{width=\maxwidth,height=\maxheight,keepaspectratio}

% Prevent slide breaks in the middle of a paragraph:
\widowpenalties 1 10000
\raggedbottom

% Comment these out if you don't want a slide with just the
% part/section/subsection/subsubsection title:
\AtBeginPart{
  \let\insertpartnumber\relax
  \let\partname\relax
  \frame{\partpage}
}
\AtBeginSection{
  \ifbibliography
  \else
    \let\insertsectionnumber\relax
    \let\sectionname\relax
    \frame{\sectionpage}
  \fi
}
\AtBeginSubsection{
  \let\insertsubsectionnumber\relax
  \let\subsectionname\relax
  \frame{\subsectionpage}
}

\setlength{\emergencystretch}{3em}  % prevent overfull lines
\providecommand{\tightlist}{%
  \setlength{\itemsep}{0pt}\setlength{\parskip}{0pt}}
\setcounter{secnumdepth}{0}

\title{SEAT de México}
\subtitle{Análisis y pronóstico de redes sociales SEAT}
\author{Data Science Area: Social Data Statistics}
\date{2016-11-23}
\usepackage{fancyhdr}
\pagestyle{fancy}
\rhead{\includegraphics[width = .075\textwidth]{Logo2a.png}}

\begin{document}
\frame{\titlepage}

\begin{frame}{Introducción}

El presente análisis tiene como objetivo dar a conocer los pronósticos
de crecimiento de las principales redes sociales de SEAT México que
involucran (Facebook, Twitter e Instagram).

Por otro lado, para el caso concreto de Facebook e Instagram, se
calcularon los pronósticos de enganche (Engagement en inglés) y
Menciones respectivamente.

Además de esto, se realizaron pronósticos sobre las visitas al sitio web
de SEAT, además de las búsquedas de concesionarias (``Dealer search'') y
``Car configurations''

\end{frame}

\begin{frame}{Origen de los datos}

\begin{itemize}
\tightlist
\item
  Facebook: \href{https://www.facebook.com/SEAT.Mexico}{SEAT México}

  \begin{itemize}
  \tightlist
  \item
    Rango de tiempo: (2016-02-01/2016-10-26)
  \item
    Periodicidad de la muestra: Diaria y mensual
  \end{itemize}
\item
  Twitter: \href{https://twitter.com/SEAT_Mexico}{SEAT\_México}

  \begin{itemize}
  \tightlist
  \item
    Rango de tiempo: (2015-12-01/2016-11-05, 2013-01-01/2016-10-23)
  \item
    Periodicidad de la muestra: Diaria
  \end{itemize}
\item
  Instagram:
  \href{https://www.instagram.com/seatmexico/?hl=es}{SEAT-México}

  \begin{itemize}
  \tightlist
  \item
    Rango de tiempo: (2014-2016 y 2015-2016)
  \item
    Periodicidad de la muestra: Diaria
  \end{itemize}
\item
  Sitio Web: \href{http://www.seat.mx/}{SEAT-México}

  \begin{itemize}
  \tightlist
  \item
    Rango de tiempo: (2015-07-01/2016-10-26 y 2015-10-01/2016-10-26)
  \item
    Periodicidad de la muestra: Diaria y mensual
  \end{itemize}
\end{itemize}

\end{frame}

\begin{frame}{Metodología}

Los datos fueron ajustados a modelos de tipo SARIMA y regresiones no
lineales que obedecen a modelos exponenciales y logísticos.

\[
SARIMA: \Phi_{P} (B^{S}) \phi(B) \nabla_{S}^{D} \nabla^{d} x_{t} = 
        \delta + \varTheta_{Q} (B^S) \theta(B) w_t
\]

\[
Logístico: \dot x = rx (1 - x/k)
\]

\[
Exponencial: P_{t} = P_{0}(1+r)^t
\]

Los cuales son modelos estadísticos que utilizan datos del pasado de una
variable en específico para posteriormente, encontrar patrones y
realizar pronósticos sobre su comportamiento en el futuro cercano.

\tiny
Todos estos análisis fueron realizados con el software estadístico R (R
Core Team 2016) en conjunto con los paquetes forecast (Hyndman 2016) y
xts(Ryan y Ulrich 2014)

\end{frame}

\section{Facebook}\label{facebook}

\begin{frame}{Facebook}

El pronóstico de crecimiento de fans online para Facebook de SEAT fue
realizado utilizando modelos SARIMA.

Para este caso se encontraron dos modelos diferentes, ambos
estadísticamente correctos, uno sugiere un crecimiento en el número de
fans online de Facebook, el otro sugiere un ligero decrecimiento en el
número de fans online para el mes de diciembre de 2016.

\end{frame}

\begin{frame}{Facebook}

\begin{block}{Fans totales}

\includegraphics{nuevos_pronosticos_seat_octubre_pdf_files/figure-beamer/unnamed-chunk-1-1.pdf}

Hasta el momento se tienen \textbf{1,603,706} fans totales, que de
acuerdo a la comparación con la cifra del último día de 2015
(\textbf{1,574,514 fans totales}) corresponde a un crecimiento de
\textbf{1.85\%}.

\end{block}

\end{frame}

\begin{frame}{Facebook}

\begin{block}{Pronóstico para fans online}

\includegraphics{nuevos_pronosticos_seat_octubre_pdf_files/figure-beamer/unnamed-chunk-2-1.pdf}

Se espera un máximo de \textbf{1,272,064} fans totales para el mes de
Diciembre de 2016. El crecimiento pronósticado es de \textbf{1.95\%} ,
actualmente se tiene un crecimiento de \textbf{0.95\%}
(\textbf{1,259,480} fans online)

\end{block}

\end{frame}

\begin{frame}{Facebook}

\begin{block}{Pronóstico de ``Engagement''}

\includegraphics{nuevos_pronosticos_seat_octubre_pdf_files/figure-beamer/unnamed-chunk-3-1.pdf}

Se espera un \textbf{1\%} de ``engagement'' para el primero de Enero de
2017

\end{block}

\end{frame}

\section{Twitter}\label{twitter}

\begin{frame}{Twitter}

\begin{block}{Pronóstico de crecimiento de seguidores}

\includegraphics{nuevos_pronosticos_seat_octubre_pdf_files/figure-beamer/unnamed-chunk-4-1.pdf}

Se espera un máximo de \textbf{302,439} seguidores para el mes de
Diciembre de 2016. El crecimiento pronósticado es de \textbf{6.63\%}. El
crecimiento actual es de \textbf{5.9\%}, es decir \textbf{300612}
seguidores.

\end{block}

\end{frame}

\begin{frame}{Twitter}

\begin{block}{Pronóstico de menciones}

\includegraphics{nuevos_pronosticos_seat_octubre_pdf_files/figure-beamer/unnamed-chunk-5-1.pdf}

\tiny
Las menciones tienen un comportamiento estable en el tiempo (media y
varianza no cambian) por lo que se espera una media de \textbf{27.4}
menciones por día con un posible total de \textbf{1918} nuevas menciones
esperadas en los próximos días del año.

\end{block}

\end{frame}

\section{Instagram}\label{instagram}

\begin{frame}{Instagram}

\includegraphics{nuevos_pronosticos_seat_octubre_pdf_files/figure-beamer/unnamed-chunk-6-1.pdf}

Este modelo fue tomado en cuenta con datos tomados desde el 31 de
Diciembre de 2015. Se espera un máximo de \textbf{31,441} seguidores
para el mes de Diciembre de 2016. El crecimiento pronósticado es de
\textbf{454\%} Actualmente el crecimiento es de \textbf{377\%}

\end{frame}

\section{Web}\label{web}

\begin{frame}{Web}

\begin{block}{Visitas}

\includegraphics{nuevos_pronosticos_seat_octubre_pdf_files/figure-beamer/unnamed-chunk-7-1.pdf}

La serie temporal de las visitas es estable en la mayoría del tiempo,
por lo que se pronostica un promedio de \textbf{19,031} visitas diarias
con un posible total de \textbf{1332170} visitas totales para el resto
de año.

\end{block}

\end{frame}

\begin{frame}{Web}

\begin{block}{Búsqueda de concesionarias}

\includegraphics{nuevos_pronosticos_seat_octubre_pdf_files/figure-beamer/unnamed-chunk-8-1.pdf}

Para el caso de ``Dealer Search'', se pronostica un promedio de
\textbf{240} búsquedas al día, es decir tiene una tendencia estable
similar a la que ocurre a partir de Julio del presente año con un total
de \textbf{20345} búsquedas en lo que resta del año.

\end{block}

\end{frame}

\begin{frame}{Web}

\begin{block}{``Car configurations''}

\includegraphics{nuevos_pronosticos_seat_octubre_pdf_files/figure-beamer/unnamed-chunk-9-1.pdf}

Para el caso de ``Car configurations'', se pronostica un promedio de
\textbf{64,441} configuraciones mensuales para lo que resta del año.

\end{block}

\end{frame}

\section{Gracias por su atención}\label{gracias-por-su-atencion}

\end{document}
